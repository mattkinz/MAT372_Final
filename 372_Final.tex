%%%%%%%%%%%%%%%%%%%%%%%%%%%%%%%%%%%%%%%%%%%%%%%%%%%%%%
% Various Solutions for Spring 17 372 class
%%%%%%%%%%%%%%%%%%%%%%%%%%%%%%%%%%%%%%%%%%%%%%%%%%%%%%
\documentclass[11pt]{article}
\usepackage{fullpage}
\usepackage{amsfonts}
\usepackage{amsmath}

\newcommand{\sss}{\mbox{$\mathcal L$}}


\begin{document}

\vspace*{-.6in}

\begin{center}
{\Large\bf Matt Kinsinger, MAT 372 Final}

\end{center}

\vspace{.5in} 

\noindent {\bf [1]} For any set $E\subseteq \mathbb{R}^2$ define the projection operator $P:E\rightarrow\mathbb{R}$ by $P(x,y)=x$.

\vspace{.2in}

\noindent (a) Suppose that $E$ is bounded. Then for all $u\in E$, there exists $M\geq 0$ such that $\|u\|\leq M$. Let $u=(u_1,u_2)\in E$. Then,
\[
|P(u)|=|P(u_1,u_2)|=|u_1|\leq \|u\|\leq M.
\]
Thus, $P(E)$ is bounded.\hfill $\Box$

\vspace{.2in}

\noindent (b) Let $E=\left\{(x,\frac{1}{x}):x>0\right\}\subseteq\mathbb{R}^2$, which is closed because any $u\in\mathbb{R}^2\cap E^c$ can be enclosed in an open ball $B(u)\subseteq E^c$, i.e. $E^c$ is open. But, $P(E)=\left\{x:x>0\right\}$ is open.\hfill $\Box$

\vspace{.2in}

\noindent (c) Suppose the $E$ is compact. Let $\bigcup\limits_{\alpha =1}^\infty G_\alpha$ be a union of open sets such that $P(E)\subseteq\bigcup\limits_{\alpha =1}^\infty G_\alpha$. It follows that
\[
E\subseteq P(E)\times\left(\bigcup\limits_{k\in\mathbb{N}}(-k,k)\right)\subseteq \left(\bigcup\limits_{\alpha =1}^\infty G_\alpha\right)\times \left(\bigcup\limits_{k\in\mathbb{N}}(-k,k)\right)
\] 
Since $E$ is compact, there exists numbers N,M such that 
\[
E\subseteq \left(\bigcup\limits_{\alpha =1}^N G_\alpha\right)\times \left(\bigcup\limits_{k=1}^M(-k,k)\right)
\]

\noindent Thus, $P(E)\subseteq \bigcup\limits_{\alpha =1}^N G_\alpha$. So $P(E)$ is compact.\hfill $\Box$

\vspace{.2in}

\noindent More succinctly (but less fun) you could use continuity of $P$ and preservation of compact sets by continuous functions.\hfill $\Box$

\vspace{.5in}

\noindent {\bf [2]} Let $B:\mathbb{R}^p\times\mathbb{R}^p\rightarrow\mathbb{R}^q$ be a bounded bilinear function. Set $g(x)=B(x,x)$. For $x,u\in\mathbb{R}^p$, prove:
\begin{align*}
\text{(i)}&\;Dg(x)(u)=B(x,u)+B(u,x)=Dg(u)(x)\\\text{(ii)}&\;g(x+u)=g(x)+g(u)+Dg(x)(u).
\end{align*}

\vspace{.2in}

\noindent Since $B$ is a bounded bilinear function, there exists $M>0$ such that $\|B(x,y)\|\leq M\|x\|\|y\|$ for all $x,y\in\mathbb{R}^p$.
\vspace{.2in}

\noindent Let $x,u\in\mathbb{R}^p$.
\vspace{.2in}

\noindent \text{(i)} To check that $F_x(u)=B(x,u)+B(u,x)$ is linear in $u$, let $u=cv+z$ for scaler $c$ and $v,z\in\mathbb{R}^p$.
\begin{eqnarray*}
F_x(cv+z)&=&B(x,cv+z)+B(cv+z,x)\\&=&cB(x,v)+B(x,z)+cB(v,x)+B(z,x)\\&=&c\left[B(x,v)+B(v,x)\right]+B(x,z)+B(z,x)\\&=&cF_x(v)+F_x(z).
\end{eqnarray*}
\noindent Thus $F_x$ is linear in $u$.

\vspace{.2in}

\noindent Let $\epsilon>0$. Choose $t\in\mathbb{R}$ such that $\|tu\|<\frac{\epsilon}{M}$. Note that
\begin{align*}
g(x+tu)&=B(x+tu,x+tu)\\&=B(x,x+tu)+tB(u,x+tu)\\&=B(x,x)+tB(x,u)+tB(u,x)+t^2B(u,u)&&(1)
\end{align*}
\noindent It follows that
\begin{align*}
\|g(x+tu)-g(x)-F_x(tu)\|&=\|g(x+tu)-g(x)-\left[B(x,tu)+B(tu,x)\right]\|\\&=\|g(x+tu)-g(x)-\left[tB(x,u)+tB(u,x)\right]\|\\&=\|B(x,x)+t^2B(u,u)-g(x)\|\\&=\|t^2B(u,u)\|\\&\leq t^2M\|u\|^2\\&=M\|tu\|\|tu\|\\&<\epsilon\|tu\|.
\end{align*}

\noindent Thus, $Dg(x)(u)=F_x(u)=B(x,u)+B(u,x)$. Using symmetry, this same argument shows that $Dg(u)(x)=B(x,u)+B(u,x)$ as well. \hfill $\Box$

\vspace{.2in}

\noindent \text{(ii)} \begin{align*}g(x+u)&=B(x+u,x+u)\\&=B(x,x+u)+B(u,x+u)\\&=B(x,x)+B(x,u)+B(u,x)+B(u,u)\\&=B(x,x)+B(u,u)+B(x,u)+B(u,x)\\&=g(x)+g(u)+Dg(x)(u).
\end{align*}
\hfill $\Box$

\vspace{.5in}

\noindent {\bf [4]} Suppose that $w\in C^2(\Omega)\cap C(\overline{\Omega})$ is a solution to the given partial differential equation. Since $\Omega\subseteq\mathbb{R}^3$ is bounded,  $\overline{\Omega}$ is both closed and bounded in $\mathbb{R}^3$, hence $\overline{\Omega}$ is compact. $w$ is continuous on a compact set, thus $w$ attains a maximum and a minimun on $\overline{\Omega}$.

\begin{itemize}
\item Case 1: Suppose that both the max and min of $w$ on $\overline{\Omega}$ occur on $\partial \Omega$. Then 
\[
\max w=\min w=0.
\]
\noindent So $w=0$ on $\overline{\Omega}$.

\item Case 2: Suppose that max $w$ occurs at $x_M\in \Omega$, and min $w$ occurs anywhere in $\overline{\Omega}$. Since $\Omega$ is open, $x_M$ is an interior point of $\Omega$. Since $w\in C^2(\Omega)$, both $Dw(x_M)$ and $D^2w(x_M)$ exist. Moreover, $Dw(x_M)y=0$ and $D^2w(x_M)y^2\leq 0$ for all $y\in \mathbb{R}^3$. This implies that 
\[
c\cdot \left(\nabla w\right)_{x_M}=c\cdot \left(\frac{\partial w}{\partial x},\frac{\partial w}{\partial y},\frac{\partial w}{\partial z}\right)_{x_M}=c\cdot \left(0,0,0\right)=0.
\]
hence
\begin{align*}
-\nabla ^2w+c\cdot \nabla w + w&=0 \\ w&=\nabla ^2w
\end{align*}

\noindent But, 
\begin{align*}
0&\geq D^2w(x_M)e_1^2=D_{11}w(x_M) \\ 0&\geq D^2w(x_M)e_2^2=D_{22}w(x_M) \\ 0&\geq D^2w(x_M)e_3^2=D_{33}w(x_M)
\end{align*}

\noindent It follows that 
\[
\max\limits_{\Omega}\;w=w(x_M)=\nabla ^2w(x_M)=D_{11}w(x_M)+D_{22}w(x_M)+D_{33}w(x_M)\leq 0
\]

\noindent Moreover, because $\Omega\subseteq \overline{\Omega}$,
\[
\min\limits_{\overline{\Omega}}\;w \leq \min\limits_{\Omega}\;w \leq \max\limits_{\Omega}\;w \leq \max\limits_{\overline{\Omega}}\;w \leq 0.\hspace{.5in}(1)
\]

\item Case 3: Suppose that min $w$ occurs at $x_m\in \Omega$, and max $w$ occurs anywhere in $\overline{\Omega}$. We can follow that same arguement in Case 1, but with
\begin{align*}
0&\leq D^2w(x_m)e_1^2=D_{11}w(x_m) \\ 0&\leq D^2w(x_m)e_2^2=D_{22}w(x_m) \\ 0&\leq D^2w(x_m)e_3^2=D_{33}w(x_m)
\end{align*}
Thus, 
\[
\min\limits_{\Omega}\;w=w(x_m)=\nabla ^2w(x_m)=D_{11}w(x_m)+D_{22}w(x_m)+D_{33}w(x_m)\geq 0
\]

\noindent Therefore, because $\Omega\subseteq \overline{\Omega}$,
\[
\max\limits_{\overline{\Omega}}\;w\geq \max\limits_\Omega\;w\geq \min\limits_\Omega\;w\geq\min\limits_{\overline{\Omega}}\geq 0.\hspace{.5in}(2)
\]

\vspace{.2in} (1) and (2) together imply that $\max\;w=0$ and $\min\;w=0$ on $\overline{\Omega}$. Thus, $w=0$ on $\overline{\Omega}$.  Since $w$ was an arbitrary solution, $w=0$ is the only solution.\hfill $\Box$


\end{itemize}


\end{document}