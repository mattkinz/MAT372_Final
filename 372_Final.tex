%%%%%%%%%%%%%%%%%%%%%%%%%%%%%%%%%%%%%%%%%%%%%%%%%%%%%%
% Various Solutions for Spring 17 372 class
%%%%%%%%%%%%%%%%%%%%%%%%%%%%%%%%%%%%%%%%%%%%%%%%%%%%%%
\documentclass[11pt]{article}
\usepackage{fullpage}
\usepackage{amsfonts}
\usepackage{amsmath}

\newcommand{\sss}{\mbox{$\mathcal L$}}


\begin{document}

\vspace*{-.6in}

\begin{center}
{\Large\bf Matt Kinsinger, MAT 372 Final}

\end{center}

\vspace{.5in} {\bf [4]} Suppose that $w\in C^2(\Omega)\cap C(\overline{\Omega})$ and $u\in C^2(\Omega)\cap C(\overline{\Omega})$ are both solutions to the given partial differential equation. Set $g=w-u$. Since $w,u\in C(\overline{\Omega})$ it follows that $g\in C(\overline{\Omega})$. Since $\Omega\subseteq\mathbb{R}^3$ is bounded,  $\overline{\Omega}$ is both closed and bounded in $\mathbb{R}^3$, hence $\overline{\Omega}$ is compact. $g$ is continuous on a compact set, thus $g$ attains a maximum and a minimun on $\overline{\Omega}$.



\end{document}