%%%%%%%%%%%%%%%%%%%%%%%%%%%%%%%%%%%%%%%%%%%%%%%%%%%%%%
% Various Solutions for Spring 17 372 class
%%%%%%%%%%%%%%%%%%%%%%%%%%%%%%%%%%%%%%%%%%%%%%%%%%%%%%
\documentclass[11pt]{article}
\usepackage{fullpage}
\usepackage{amsfonts}
\usepackage{amsmath}

\newcommand{\sss}{\mbox{$\mathcal L$}}


\begin{document}

\vspace*{-.6in}

\begin{center}
{\Large\bf Matt Kinsinger, MAT 372 Final}

\end{center}

\vspace{.5in} {\bf [4]} Suppose that $w\in C^2(\Omega)\cap C(\overline{\Omega})$ is a solution to the given partial differential equation. Since $\Omega\subseteq\mathbb{R}^3$ is bounded,  $\overline{\Omega}$ is both closed and bounded in $\mathbb{R}^3$, hence $\overline{\Omega}$ is compact. $w$ is continuous on a compact set, thus $w$ attains a maximum and a minimun on $\overline{\Omega}$.

\begin{itemize}
\item Case 1: Suppose that both the max and min of $w$ on $\overline{\Omega}$ occur on $\partial \Omega$. Then 
\[
\max w=\min w=0.
\]
\noindent So $w=0$ on $\overline{\Omega}$.

\item Case 2: Suppose that max $w$ occurs at $x_M\in \Omega$, and min $w$ occurs anywhere in $\overline{\Omega}$. Since $\Omega$ is open, $x_M$ is an interior point of $\Omega$. Since $w\in C^2(\Omega)$, both $Dw(x_M)$ and $D^2w(x_M)$ exist. Moreover, $Dw(x_M)y=0$ and $D^2w(x_M)y^2\leq 0$ for all $y\in \mathbb{R}^3$. This implies that 
\[
c\cdot \left(\nabla w\right)_{x_M}=c\cdot \left(\frac{\partial w}{\partial x},\frac{\partial w}{\partial y},\frac{\partial w}{\partial z}\right)_{x_M}=c\cdot \left(0,0,0\right)=0.
\]
hence
\begin{align*}
-\nabla ^2w+c\cdot \nabla w + w&=0 \\ w&=\nabla ^2w
\end{align*}

\noindent But, 
\begin{align*}
0&\geq D^2w(x_M)e_1^2=D_{11}w(x_M) \\ 0&\geq D^2w(x_M)e_2^2=D_{22}w(x_M) \\ 0&\geq D^2w(x_M)e_3^2=D_{33}w(x_M)
\end{align*}

\noindent It follows that 
\[
\max\limits_{\Omega}\;w=w(x_M)=\nabla ^2w(x_M)=D_{11}w(x_M)+D_{22}w(x_M)+D_{33}w(x_M)\leq 0
\]

\noindent Moreover, because $\Omega\subseteq \overline{\Omega}$,
\[
\min\limits_{\overline{\Omega}}\;w \leq \min\limits_{\Omega}\;w \leq \max\limits_{\Omega}\;w \leq \max\limits_{\overline{\Omega}}\;w \leq 0.\hspace{.5in}(1)
\]

\item Case 3: Suppose that min $w$ occurs at $x_m\in \Omega$, and max $w$ occurs anywhere in $\overline{\Omega}$. We can follow that same arguement in Case 1, but with
\begin{align*}
0&\leq D^2w(x_m)e_1^2=D_{11}w(x_m) \\ 0&\leq D^2w(x_m)e_2^2=D_{22}w(x_m) \\ 0&\leq D^2w(x_m)e_3^2=D_{33}w(x_m)
\end{align*}
Thus, 
\[
\min\limits_{\Omega}\;w=w(x_m)=\nabla ^2w(x_m)=D_{11}w(x_m)+D_{22}w(x_m)+D_{33}w(x_m)\geq 0
\]

\noindent Therefore, because $\Omega\subseteq \overline{\Omega}$,
\[
\max\limits_{\overline{\Omega}}\;w\geq \max\limits_\Omega\;w\geq \min\limits_\Omega\;w\geq\min\limits_{\overline{\Omega}}\geq 0.\hspace{.5in}(2)
\]

\vspace{.2in} (1) and (2) together imply that $\max\;w=0$ and $\min\;w=0$ on $\overline{\Omega}$. Thus, $w=0$ on $\overline{\Omega}$.  Since $w$ was an arbitrary solution, $w=0$ is the only solution.\hfill $\Box$


\end{itemize}


\end{document}