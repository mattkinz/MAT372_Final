%%%%%%%%%%%%%%%%%%%%%%%%%%%%%%%%%%%%%%%%%%%%%%%%%%%%%%
% Various Solutions for Spring 17 372 class
%%%%%%%%%%%%%%%%%%%%%%%%%%%%%%%%%%%%%%%%%%%%%%%%%%%%%%
\documentclass[11pt]{article}
\usepackage{fullpage}
\usepackage{amsfonts}
\usepackage{amsmath}

\newcommand{\sss}{\mbox{$\mathcal L$}}


\begin{document}

\vspace*{-.6in}

\begin{center}
{\Large\bf Matt Kinsinger, MAT 372 Final}

\end{center}

\vspace{.5in} {\bf [4]} Suppose that $w\in C^2(\Omega)\cap C(\overline{\Omega})$ is a solution to the given partial differential equation. Since $\Omega\subseteq\mathbb{R}^3$ is bounded,  $\overline{\Omega}$ is both closed and bounded in $\mathbb{R}^3$, hence $\overline{\Omega}$ is compact. $w$ is continuous on a compact set, thus $w$ attains a maximum and a minimun on $\overline{\Omega}$.

\begin{itemize}
\item Case 1: Suppose that both the max and min of $w$ on $\overline{\Omega}$ occur on $\partial \Omega$. Then 
\[
\max w=\min w=0.
\]
\noindent So $w=0$ on $\overline{\Omega}$.

\item Case 2: Suppose that max $w$ occurs at $x_M\in \Omega$. $x_M$ is an interior point of $\Omega$ since $\Omega$ is open.  


\end{itemize}


\end{document}